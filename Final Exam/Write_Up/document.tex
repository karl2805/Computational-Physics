\documentclass[11pt]{article}
\usepackage[a4paper, total={7.5in, 10.5in}]{geometry}
\usepackage{amsmath}
\usepackage{graphicx}
\usepackage{siunitx}
\usepackage{wrapfig}
\usepackage{subcaption}
\usepackage{gensymb}
\usepackage{caption}
\usepackage[export]{adjustbox}
\usepackage[hidelinks]{hyperref}
\setlength\fboxsep{10pt}
\setlength\fboxrule{1pt}
\usepackage{wrapfig}
\hbadness = 1000000000
\usepackage{pythonhighlight}
\setlength{\parindent}{0pt}
\usepackage{booktabs}
\usepackage{float}
\usepackage{multicol}
\usepackage[thinc]{esdiff}
\usepackage{minted} 
\usepackage{siunitx}
\newcommand{\solarmass}{\textup{M}_\odot}
\newcommand{\earthmass}{\textup{M}_{E}}
\newcommand{\jupitermass}{\textup{M}_{J}}
\newcommand{\uvec}[1]{\boldsymbol{\hat{\textbf{#1}}}}
\newcommand{\vvec}[1]{\vec{\boldsymbol{#1}}}
\usepackage{amsmath}
\usepackage{calc}
\usepackage{hyperref}
\hypersetup{colorlinks=false}
\makeatletter
\newcommand*\@dblLabelI {}
\newcommand*\@dblLabelII {}
\newcommand*\@dblequationAux {}

\def\@dblequationAux #1,#2,%
{\def\@dblLabelI{\label{#1}}\def\@dblLabelII{\label{#2}}}
\newcommand*{\doubleequation}[3][]{%
	\par\vskip\abovedisplayskip\noindent
	\if\relax\detokenize{#1}\relax
	\let\@dblLabelI\@empty
	\let\@dblLabelII\@empty
	\else % we assume here that the optional argument
	% has the required shape A,B
	\@dblequationAux #1,%
	\fi
	\makebox[0.5\linewidth-1.5em]{%
		\hspace{\stretch2}%
		\makebox[0pt]{$\displaystyle #2$}%
		\hspace{\stretch1}%
	}%
	\makebox[0.5\linewidth-1.5em]{%
		\hspace{\stretch1}%
		\makebox[0pt]{$\displaystyle #3$}%
		\hspace{\stretch2}%
	}%
	\makebox[3em][r]{(%
		\refstepcounter{equation}\theequation\@dblLabelI, 
		\refstepcounter{equation}\theequation\@dblLabelII)}%
	\par\vskip\belowdisplayskip
}
\newcommand*{\ndoubleequation}[3][]{%
	\par\vskip\abovedisplayskip\noindent
	\if\relax\detokenize{#1}\relax

	\else % we assume here that the optional argument
	% has the required shape A,B
	\@dblequationAux #1,%
	\fi
	\makebox[0.5\linewidth-1.5em]{%
		\hspace{\stretch2}%
		\makebox[0pt]{$\displaystyle #2$}%
		\hspace{\stretch1}%
	}%
	\makebox[0.5\linewidth-1.5em]{%
		\hspace{\stretch1}%
		\makebox[0pt]{$\displaystyle #3$}%
		\hspace{\stretch2}%
	}%
\par\vskip\belowdisplayskip
	
}




\begin{document}





\begin{center}
	\huge \textbf{\underline{Orbital Integration of Earth and Jupiter}} \\
	\vspace{10pt}
	\large Karl Riedewald 123377661\\
	
\end{center}

\vspace{70pt}

\section{Introduction}
This report compares the accuracy and stability of the Euler and Runge-Kutta second and fourth order methods to solve differential equations numerically. Each method was applied to solve the system of the Earth's orbit around the Sun. The method of greatest accuracy was then applied to the more complex system of the Earth and Jupiter orbiting the Sun and the affect of Jupiter's influence on the Earth's orbit was examined.\\

Numerical methods find approximate solutions to differential equations by through iterative calculations based on a time step $dt$. These methods use the derivative at each time step to estimate the value at the next point which is then used to calculate the next derivative and the process repeats. \\

The accuracy of a particular method depends on how the derivative is used to calculate the new value as well as the size of the iterative time step. Small errors in each iterative calculation can accumulate over time to lead to quite inaccurate solutions over time. The accuracy of each method can be compared by calculating the total energy and angular momentum of the simulation at each time step, something which does not change in a physically isolated system such as the Earth orbiting the Sun.\\

The Earth orbits the Sun with a period of approximately 365 days with a mean orbital radius defined as 1 Astronomical Unit (AU). The orbit of the Earth around the Sun is slightly elliptical with an apoapsis of 1.01 AU and a periapsis of 0.98 AU. In the simulation presented here, the Earth's orbit was assumed circular with a radius of 1 AU and a velocity given by the Keplerian velocity. In addition, the Sun's mass was assumed to be sufficiently large relative to the Earth such that its motion can be neglected.\\

Jupiter has by far the greatest mass in our solar system, with a mass of more than two and half times that of all the other planets combined, about 317 times that of the Earth. Jupiter completes an orbit of the Sun about every 12 years with an average orbital radius of about 5.2 AU. Even at this great distance, the sheer mass of Jupiter causes its gravitational pull on the Earth to be significant factor of the Earth's orbit around the Sun. Using the integration method of greatest accuracy, Jupiter was added to the Earth orbit simulation and its affect discussed.

\newpage

\section{Orbit of Earth Around the Sun}
\subsection{Mathematical Analysis}
From Newton's universal law of gravitation, the gravitational force $F_{E}$ acting on the Earth as it orbits the Sun is given by

\begin{equation}
	F_{E} = -\frac{G\earthmass\solarmass}{r^{2}}\uvec{r}\\
	\label{eq:Newton's Law of Gravitation}
\end{equation}

where $G$ is the universal gravitational constant, $\solarmass$ the mass of the sun, $\earthmass$ the mass of the earth, $r$ the distance between the Earth and the Sun and $\uvec{r}$ the unit vector pointing from the Earth to the Sun. The acceleration $a_{E}$ that the earth experiences at any time during its orbit around the Sun is then given from Newton's second law of motion by

\begin{equation}
	a_{E} = \frac{F_{E}}{\earthmass} = -\frac{G\solarmass}{r^{2}}\uvec{r} = -\frac{G\solarmass\vvec{r}}{r^3}\\
	\label{eq:Accelleration of the Earth}
\end{equation} 

where $\vvec{r}$ is the vector pointing from the Earth to the Sun.\\

It is now assumed that the mass of the Sun is sufficiently large relative to that of the Earth such that its motion can be ignored. Setting the Sun's position to the centre of the Cartesian coordinate system, the x and y components of the Earth's position as it orbits the Sun can now be written as

\doubleequation[eq:accelx,eq:accely]{\diff[2]{x}{t} = -\frac{G\solarmass x}{r^3}}{\diff[2]{y}{t} = -\frac{G\solarmass y}{r^3}}

Now in order to solve these differential equations numerically, they need to be split into two separate first order differential equations. By setting the velocities in the x and y directions to

\doubleequation[eq:vx,eq:vy]{\diff{x}{t} = v_{x}}{\diff{y}{t} = v_{y}}


equations \ref{eq:accelx} and \ref{eq:accely} can be rewritten in therms of the acceleration $a_{x}$ and $a_{y}$ as

\doubleequation[eq:ax,eq:ay]{a_{x} = \diff{v_{x}}{t} = -\frac{G\solarmass x}{r^3}}{a_{y} = \diff{v_{y}}{t} =  -\frac{G\solarmass y}{r^3}}

This set of four differential equations are now all of first order and can thus be solved using the numerical integration methods.

\newpage

\subsection{Euler Method}
\subsubsection{Implementation}
The Euler method is a simple technique which the next value $y_{i+1}$ of first order differential equations over step $h$ of the form

\begin{equation}
	\diff{y}{x} = f(x, y),\, \, \, \,  y(0) = y_{0}
	\label{eq:diff_form}
\end{equation}

using the following equation 

\begin{equation}
	y_{i+1} = y_{i} + f(x_{i}, y_{i})h\\
	\label{eq:Euler Method}
\end{equation}

\vspace{10pt}
In order to solve the above differential equations with C\texttt{++} in a nice manner using the Euler method, a state vector was defined as an array which stores the Earth's $x, y$ position in the first two indexes and the Earth's velocity components $v_{x}, v_{y}$ in the last two indexes. A function `Earth Orbit' was then defined which takes in the current state vector and returns the derivative of the state vector, namely the velocity $v_{x}, v_{y}$ and acceleration $a_{x}, a_{y}$. (This function corresponds to the $f(x_{i}, y_{i})$ function of equation \ref{eq:Euler Method}). The acceleration is calculated from equation \ref{eq:ax} and \ref{eq:ay}. The new velocities can simply be passed on from the old state vector due to the nature of equations \ref{eq:vx} and \ref{eq:vy}. Below is the code of this function.

%\begin{verbatim}
\begin{minted}{cpp}
typedef std::array<double, 4> State; 

State EarthOrbit(State s) 
{
	//calculate the magnitude of r to find acceleration later
	double r = std::sqrt(s[0] * s[0] + s[1] * s[1]);
	
	//calcualte the acceleration for the x and y components
	double a_x = (-G * M_Sun * s[0] ) / (r * r * r);
	double a_y = (-G * M_Sun * s[1] ) / (r * r * r);
	
	//return the derivative state using the velocities of the input state vector
	return {s[2], s[3], a_x, a_y};
}	
\end{minted}
%\end{verbatim}


The function `EulerStep' was then defined which takes in a function (the `EarthOrbit' function in this case) and simply implements equation \ref{eq:Euler Method} as show here

%\begin{verbatim}
\begin{minted}{cpp}
State EulerStep(State (*f)(State s), State& s, double dt)
{
	return s + f(s) * dt;
}	
\end{minted}
%\end{verbatim}

This function with a specified step size dt was then run in a loop over a certain time frame to numerically solve the required differential equations using the Euler method.


\newpage


\subsubsection{Results}

The Euler integration method was run for the Earth around the Sun for a simulated time of 5 years for three different step sizes of a day, a half day and a quarter of a day as shown in figure \ref{gr:Euler_Earth}. The initial $x$ position was set to 1 Astronomical Unit (AU) while the initial $y$ position was set to 0. The initial velocity of the Earth was set to the Keplerian velocity in the positive y direction given by

\begin{equation}
	v_{y_{0}} = \sqrt{\frac{G\earthmass}{r}}
	\label{eq:Kepler}
\end{equation}

The initial velocity in the x direction $v_{x_{0}}$ was set to zero.\\

\begin{figure}[h]
	\centering
	\includegraphics[scale=0.5]{Euler_Orbit}
	\caption{The Euler integration method performed for 3 different step sizes over a simulated time of 5 years. The Earth's initial position was set to 1 AU to the right of Sun which is represented by the yellow circle in each plot. The initial velocity was set the the Keplerian velocity}
	\label{gr:Euler_Earth}
\end{figure}

As can be seen from the plots of figure \ref{gr:Euler_Earth}, the Euler method is not very stable. The radius of Earth's orbit increases dramatically over the 5 year simulation time. While decreasing the step size does decrease the rate at which the radius of the orbit increases, the orbit is still unstable even for a step size as low as a quarter of a day. It can also be seen from these graphs that 

\newpage

In reality, in the case where the earth is the only planet orbiting the Sun and assuming the Sun is fixed in place at the origin, the magnitude of the angular momentum of the earth and the total energy of the system would be conserved.\\

The total energy of the system is just the total energy of the Earth as the position of the Sun is assumed to be fixed. The total energy of the earth is then given by the its total kinetic and potential energy. The potential $U_{E}$ and kinetic $K_{E}$ energy of the earth are given by
\begin{equation}
	U_{E} = -\frac{G\solarmass\earthmass}{r} \hspace{50pt} K_{E} = \frac{1}{2}\earthmass v^{2}
\end{equation} 

Thus the total energy $E$ of the earth at a certain point can be calculated as 

\begin{equation}
	E = \frac{1}{2}\earthmass v^{2} -\frac{G\solarmass\earthmass}{r}
	\label{eq:Earth_Energy}
\end{equation}

The angular momentum $L$ of the Earth is given by
\begin{equation}
	\boldmath{L} = \boldmath{r} \times \boldmath{p}
	\label{eq:Earth_L}
\end{equation}

Where $\boldmath{p}$ is the instantaneous linear momentum of the Earth given by

\begin{equation*}
	\boldmath{p} = v\earthmass
\end{equation*}
where $v$ is the velocity of the Earth. Calculations of equations \ref{eq:Earth_Energy} and \ref{eq:Earth_L} were performed and saved after each Euler step with the plotted results shown in figure \ref{gr:Energy_L_Euler}

\begin{figure}[h]
	\centering
	\includegraphics[scale=0.6]{Euler_E_L}
	\caption{Plots of the total energy and angular momentum of the system with time for the Euler method over a simulated time of 5 years with a step size of a day.}
	\label{gr:Energy_L_Euler}
\end{figure}
Both the total energy and angular momentum of the system are seen to be rising quite rapidly which again highlights the instability and inaccuracy of the Euler method.

\newpage

\subsection{Runge-Kutta Methods}
The Runge-Kutta methods are a set of more sophisticated numerical technique compared with the Euler method. The Runge-Kutta 2nd order (RK2) and 4th order (RK4) methods will be discussed here.

\subsubsection{RK2 (Ralston's method)}
The Ralston RK2 method is given by the following equation to solve differential equations again of the form given by equation \ref{eq:diff_form}
\begin{equation}
	y_{i+1} = y_{i} + \left(\frac{1}{3}k_{1} + \frac{2}{3}k_{2}\right) h
\end{equation}

where $k_{1}$ and $k_{2}$ are given by

\begin{equation*}
	k_{1} = f(x_{i}, y_{i}) \hspace{30pt} k_{2} = f\left(x_{i} + \frac{3}{4}h, y_{i} + \frac{3}{4}k_{1}h\right)
\end{equation*}

To implement this using state vectors, the RK2 step function becomes
%\begin{verbatim}
\begin{minted}{cpp}
State RK2StepRalston(State (*f)(State s), State& s, double dt)
{
	State k1 = f(s);
	State k2 = f(s + 0.25 * k1 * dt);
	
	//perform rk2 step
	return s + ((1.0/3.0) * k1 + (2.0/3.0) * k2) * dt;
}	
\end{minted}
%\end{verbatim}

with the `EarthOrbit' function from before passed in.

\subsubsection{RK4 Method}
Similarly the RK4 method is given by 

\begin{equation}
	y_{i+1} = y_{i} + \frac{1}{6}(k_{1} + 2k_{2} + 2k_{3} + k_{4})h
\end{equation}

where 

\begin{align*}
	k_{1} &= f(x_{i}, y_{i})\\
	k_{2} &= f\left(x_{i} + \frac{1}{2}h, y_{i} + \frac{1}{2}k_{1}h\right)\\
	k_{3} &= f\left(x_{i} + \frac{1}{2}h, y_{i} + \frac{1}{2}k_{2}h\right)\\
	k_{4} &= f(x_{i} + h, y_{i} + k_{3}h)\\
\end{align*}

The Rk4 step function is then written as 
%\begin{verbatim}
\begin{minted}{cpp}
State RK4Step(State (*f)(State s), State& s, double dt)
{
	State k1 = f(s);
	State k2 = f(s + 0.5 * k1 * dt);
	State k3 = f(s + 0.5 * k2 * dt);
	State k4 = f(s + k3 * dt);
	
	//perform rk4 step
	return s + (1.0 / 6.0) * (k1 + 2 * k2 + 2 * k3 + k4) * dt;
}	
\end{minted}
%\end{verbatim}

\subsubsection{Results}

\begin{figure}[H]
	\centering
	\includegraphics[scale=0.35]{RK2_Orbit}
	\caption{Earth's orbit simulation using the RK2 method over a simulation time of 5 years with three different step sizes. This method is still unstable for these time steps as can be seen, but is slightly better than the previous Euler method as will be discussed later.}
	\label{gr:RK2_Orbit}
\end{figure}

\begin{figure}[H]
	\centering
	\includegraphics[scale=0.35]{RK4_Orbit}
	\caption{Earth's orbit simulation using the RK4 method over a simulation time of 5 years with three different step sizes. This is the first method to give stable orbits for each time step.}
	\label{RK4_Orbit}
\end{figure}

In order to compare the orbital evolution of each method presented, the radius of the orbit as it evolves with time is given in figure \ref{gr:Radius};
\begin{figure}[H]
	\centering
	\includegraphics[scale=0.5]{Radius}
	\caption{The evolution of the radius of the orbit of Earth with time for each integration method.}
	\label{gr:Radius}
\end{figure}

Figure \ref{gr:Radius} makes it clear that the radius of the orbit is increasing the fastest for the Euler method with the RK2 method radius increasing slightly slower. The RK4 method is stable over the 5 year timespan with the radius staying fixed, indicating a circular orbit, which is the expected behaviour of this two body system with the specified initial conditions. It is interesting to note that the other two methods are adding an elliptical shape to the orbit which increases with the radius increasing as is seen by the wavy nature of their plots in figure \ref{gr:Radius}.



\begin{figure}[h]
	\centering
	\includegraphics[scale=0.6]{RK2_E_L}
	\caption{The energy and angular momentum evolution of the Earth for each method over a timespan of 5 years with a step size of a day.}
	\label{gr:Comparing_E_L}
\end{figure}

Figure \ref{gr:Comparing_E_L} shows the energy and angular momentum evolution of the earth for each integration method. In a similar way to the figure \ref{gr:Radius}, using the Euler method, the Earth gains energy and angular momentum the fastest with RK2 gaining the two quantities at a slower rate. The RK4 method again shows stability with the energy and angular momentum of the Earth not changing at all over the course of the 5 year simulation. \\

From these comparisons, it is clear that the RK4 method shows by far the most stability and accuracy and will therefore be used for the simulation of the next section.

\newpage

\section{Orbit of Earth and Jupiter}
The planet Jupiter will now be added to the previous simulation and its affects on the orbit of Earth determined.\\

To model the Earth-Jupiter system, four differential equations similar to those given by equations \ref{eq:accelx} and \ref{eq:accely} will need to be solved, one for each of the x, y coordinates of the Earth and Jupiter. However, in order to model this system accurately, the differential equations will need to be modified to account for the force present between the Earth and Jupiter. \\

The force $F_{EJ}$ on the Earth from Jupiter is given by a modified version of equation 1

\begin{equation}
	F_{EJ} = -\frac{G\earthmass \jupitermass}{d^{2}}\uvec{d} = -\frac{G\earthmass \jupitermass \vvec{d}}{d^{3}}
	\label{eq:force_earth_jupiter}
\end{equation}

where $d$ is the distance between the Earth and Jupiter , $\jupitermass$ the mass of Jupiter and $\uvec{d}$ is the unit vector pointing from the Earth to Jupiter and $\vvec{d}$ the position vector from the Earth to Jupiter. With the inclusion of Jupiter, equations \ref{eq:accelx}  and \ref{eq:accely} for the Earth position $x_{E}, y_{E}$ and solar radius $r_{E}$then become

\ndoubleequation[eq:accelx2E,eq:accely2E]{\diff[2]{x_{E}}{t} = -\frac{G\solarmass x_{E}}{r_{E}^3} - \frac{G\jupitermass \vvec{d_{x}}}{d^3}}{\hspace{50pt}\diff[2]{y_{E}}{t} = -\frac{G\solarmass y_{E}}{r_{E}^3} -\frac{G\jupitermass \vvec{d_{y}}}{d^3}}

and for the Jupiter position $x_{J}, y_{J}$ and solar radius $r_{J}$

\ndoubleequation[eq:accelx2J,eq:accely2J]{\diff[2]{x_{J}}{t} = -\frac{G\solarmass x_{J}}{r_{J}^3} + \frac{G\earthmass \vvec{d_{x}}}{d^3}}{\hspace{50pt}\diff[2]{y_{J}}{t} = -\frac{G\solarmass y_{J}}{r_{J}^3} + \frac{G\earthmass \vvec{d_{y}}}{d^3}}

Transforming these four second order differential equations into first order differential equations to solve numerically gives

\vspace{30pt}

\ndoubleequation[eq:vxE,eq:vyE]{\diff{x_{E}}{t} = v_{x_{E}}}{\diff{y_{E}}{t} = v_{y_{E}}}

\ndoubleequation[eq:axE,eq:ayE]{a_{x_{E}} = \diff{v_{x_{E}}}{t} = -\frac{G\solarmass x_{E}}{r_{E}^3} - \frac{G\jupitermass \vvec{d_{x}}}{d^3}}{\hspace{40pt}a_{y_{E}} = \diff{v_{y}}{t} =  -\frac{G\solarmass y_{E}}{r_{E}^3} - \frac{G\jupitermass \vvec{d_{y}}}{d^3}}

\vspace{30pt}

\ndoubleequation[eq:vxJ,eq:vyJ]{\diff{x_{J}}{t} = v_{x_{J}}}{\diff{y_{J}}{t} = v_{y_{J}}}

\ndoubleequation[eq:axE,eq:ayE]{a_{x_{J}} = \diff{v_{x_{J}}}{t} = -\frac{G\solarmass x_{J}}{r_{J}^3} + \frac{G\jupitermass \vvec{d_{x}}}{d^3}}{\hspace{40pt} a_{y_{J}} = \diff{v_{y_{J}}}{t} =  -\frac{G\solarmass y_{J}}{r_{J}^3} + \frac{G\jupitermass \vvec{d_{y}}}{d^3}}

\vspace{20pt}

To implement this set of differential equations, a new state vector `Two\_State' was declared which holds 8 variables, namely the $x, y$ position and the velocity components $v_{x}, v_{y}$ of both the Earth and Jupiter. The function `JupiterEarthSystem' was then declared which implements the above differential equations outputting the new state vector, similarly to the `EarthOrbit' function. An RK4 step function, identical to the previous step function except to handle Two\_State vectors was declared. Running this step function in a loop then numerically solves this set of differential equations.

\newpage

\subsection{Results}
The simulation was run for a simulated 12 years with a time step of a day. The initial positions of the Earth and Jupiter were set as as follows

\begin{align*}
	x_{0_{E}} &= 1 \, \text{AU}\\
	y_{0_{E}} &= 0 \\[10pt]
	x_{0_{J}} &= -5.2 \, \text{AU} \\
	y_{0_{J}} &= 0 
\end{align*}

As in the previous section, the initial velocity of the Earth was set to the Keplarian velocity given by equation \ref{eq:Kepler}. The initial velocity of Jupiter was also set to the Keplarian velocity but in the negative y direction to ensure that both planets orbit in the anticlockwise direction as is the case in reality.

\begin{figure}[H]
	\centering
	\includegraphics[scale=0.7]{Earth_Jupiter_Orbit}
	\caption{The simulated orbit of both the Earth and Jupiter over a 24 year timescale with a step size of a day using the RK4 method.}
	\label{gr:earth_and_jupiter}
\end{figure}

At the scale of figure \ref{gr:earth_and_jupiter}, the orbit of the Earth seems unaffected by the presence of Jupiter. In order to show Jupiter's affect, the radius of the Earth's orbit is plotted over time.

\begin{figure}[H]
	\centering
	\includegraphics[scale=0.5]{Earth_Radius}
	\caption{The radius of the Earth's orbit around the Sun over a 24 year timespan with the presence of Jupiter's orbit. The red lines denote the points where Jupiter completes one full orbit (12 years).}
	\label{gr:earth radius}
\end{figure}

Figure \ref{gr:earth radius} shows the orbit of the Earth becoming more and less elliptical with a beating pattern period of 12 years, the period of Jupiter's orbit. \\



\begin{figure}[h]
	\centering
	\includegraphics[scale=0.5]{Earth_Energy}
	\caption{The change in the total mechanical energy of Earth from the initial mechanical energy with time over a timespan of 24 years. }
	\label{gr:Earth_Energy}
\end{figure}

Figure 9 shows the change in the total mechanical energy of the earth from its initial total mechanical energy with time. The change in energy was plotted as it was found that plotting the total energy with time showed artefacts in the plot due to the difference being too small for the plotting programme. This energy was calculated using equation \ref{eq:Earth_Energy} with the addition of the half the potential energy between the Earth and Jupiter given by

\begin{equation}
	U_{EJ} = -\frac{G\earthmass\jupitermass}{d}
\end{equation}

$d$ being the distance between Earth and Jupiter. The other half of this potential was added to the energy calculation for Jupiter.


\newpage

\begin{figure}[h]
	\centering
	\includegraphics[scale=0.5]{Earth_Angular_Momentum}
	\caption{The change in the angular momentum of Earth from the initial angular momentum with time over a timespan of 24 years.}
	\label{gr:Earth L}
\end{figure}

To calculate the angular momentum of the Earth at a certain position, the position of the centre of mass $\boldmath{R}$ was found by the following equation

\begin{equation}
	\boldmath{R} = \frac{\earthmass r_{E} + \jupitermass r_{J}}{\earthmass + \jupitermass + \solarmass} 
	\label{eq:com}
\end{equation}

This is assuming that the position of the Sun is fixed at the origin and hence $r_{\text{sun}} = 0$. The angular momentum $L$ was then calculated from 

\begin{equation}
	L = (\boldmath{R} - \boldmath{r_{E})} \times \boldmath{p}
\end{equation}

where $\boldmath{p}$ is the linear momentum of the Earth. 




\newpage




\begin{figure}[H]
	\centering
	\includegraphics[scale=0.5]{Jupiter_Energy}
	\caption{The change in the total mechanical energy of Jupiter from the initial mechanical energy with time over a timespan of 24 years.}
	\label{gr:Jupiter energy}
\end{figure}



\begin{figure}[H]
	\centering
	\includegraphics[scale=0.5]{Jupiter_Angular_Momentum}
	\caption{The change in the angular momentum of Jupiter from the initial angular momentum with time over a timespan of 24 years.}
	\label{gr:Jupiter L}
\end{figure}

The total mechanical energy and angular momentum of Jupiter was calculated in the same way as the Earth with the relevant substitutions for Jupiter.\\

As Jupiter is much more massive than the Earth, it is expected that the percentage change in Jupiter's energy and angular momentum be much smaller than that of the Earth. From figure \ref{gr:Earth_Energy}, the change in the energy of the Earth was found to be approximately $\Delta E \approx -6.2 \times 10^{29} \unit{J}$. The initial total energy of the Earth was found to be $E_{0} = -2.56 \times 10 ^{33} \unit{J}$. This gives a percentage change $\delta E_{\text{Earth}}$ of 

\begin{align*}
	\delta E_{\text{Earth}} =  \frac{\Delta E}{E_{0}} = 2.4 \times 10 ^{-3} \% 
\end{align*}

Applying the same to Jupiter, the percentage change to Jupiter's total mechanical energy was found to be

\begin{align}
	\delta E_{\text{Jupiter}} = 9.88 \times 10^{-5} \%
\end{align}

Thus as expected the change to Jupiter's energy was much smaller than that of the Earth due to the much greater mass of Jupiter. A similar result was found for the angular momenta.












\end{document}
